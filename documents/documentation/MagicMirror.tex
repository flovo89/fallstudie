\documentclass[a4paper, 11pt]{article}

% deutsche Silbentrennung
\usepackage[ngerman]{babel}

% wegen deutschen Umlauten
\usepackage[]{inputenc}
\usepackage[T1]{fontenc}

\usepackage{graphicx}

\usepackage[backend=bibtex,style=trad-plain]{biblatex}
\bibliography{Bibliography.bib}

% Title Page
\title{Magic Mirror}
\author{Florian Vogel}

\begin{document}
	
\maketitle

\newpage

\tableofcontents

\newpage

\section{Einf�hrung}
Ein Magic Mirror ist ein optisch ansprechendes Anzeigeger�t. Es handelt sich um ein Spiegel mit integriertem Bildschirm, wobei es sich bei dem Spiegel um einen sogenannten Spionspiegel handelt. Er ist von einer Seite m�glichst reflektierend und von der anderen Seite m�glichst durchl�ssig.
Mit dem verbauten Bildschirm ergeben sich beinahe unbegrenzte M�glichkeiten um Informationen zu pr�sentieren und diese ansprechend darzustellen. Dadurch passt ein Magic Mirror mit passendem Design gut in einen Wohnbereich.
\\Nun, was soll denn auf solch einem Spiegel angezeigt werden? Nat�rlich gibt es einige Klassiker, wie zum Beispiel die aktuelle Zeit. Die M�glichkeiten lassen aber viel mehr zu. Es ist beispielsweise auch denkbar jeweils den n�chsten Zug von Bern nach Z�rich auf dem Magic Mirror anzuzeigen. 
\\Die Interessen f�r Informationen werden sich mit Sicherheit �ndern �ber die Zeit. Das bedingt eine Konfigurationsm�glichkeit f�r den Benutzer des Spiegels, mit welcher er anzeigende Informationen �ndern kann. Diese genannte Modularit�t zu erreichen ist eines der Ziele in dieser Semesterarbeit. Weiter soll am Ende ein funktionierender Prototyp eines Magic Mirrors vollendet sein, welcher im Heimbereich eingesetzt wird.

\newpage

\section{Zielsetzung}
Das erste und am h�chsten gewichtete Ziel ist das Erstellen eines fertigen Prototypes. Dies wird in folgende Teilschritte unterteilt.
\begin{itemize}
	\item Einfache Anzeigeelemente auf dem Spiegel, wie beispielsweise die Uhrzeit, das Wetter oder einen Kalender
	\item Energiesparmodus, dabei wird der Bildschirm �ber einen externen Infrarotsensor ein- und ausgeschalten.
	\item Modul auf welches �ber mobile Applikation zugegriffen werden kann.
	\item Eine Android Applikation um auf obengenanntes Modul zugreifen zu k�nnen.
	\item Sprachsteuerung welche vom Google Assistant gemacht wird.
\end{itemize}
Die genannte Punkte werden schrittweise umgesetzt. Somit hat der letzte Punkt die niedrigste Priorit�t und wird als optional erachtet.

\newpage

\section{Ausgangslage}
Die Idee des Magic Mirror ist bekannt. Es gibt bereits zahlreiche Versionen davon online zum Nachbau. Es gibt Vorschl�ge f�r das Spiegelglas, den Bildschirm, die Recheneinheit, sogar wie der Rahmen des Spiegels aufgebaut werden kann. Deshalb ist es grunds�tzlich einmal notwendig, in diesem Dschungel von Ideen eine gute Zusammensetzung zu finden.
\\Sehr prominent tritt dabei ein Open Source Projekt auf, welches sich MagicMirror\textsuperscript{2} nennt.
\cite{Doe:2009:Online}


\newpage

\section{Aufbau}

\subsection{Auswahl Framework}
\subsection{Auswahl Hardware}
\subsection{Auswahl Software}

\newpage

\printbibliography


\end{document}          
